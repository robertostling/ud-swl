\documentclass[final]{beamer}
\mode<presentation>
{
        \usetheme{SU}
}
\usepackage{graphicx}
\usepackage[english]{babel}
\usepackage[utf8]{inputenc}
\usepackage[orientation=portrait,size=a1,scale=1.4]{beamerposter}
\usepackage{tikz-dependency}

\usepackage[T1, OT1]{fontenc}
\DeclareTextSymbolDefault{\dh}{T1}


\title{Universal Dependencies for Swedish Sign Language}
\author{Robert {\"O}stling, Carl B{\"o}rstell, Moa G{\"a}rdenfors,
    Mats Wir{\'e}n}
\institute{Stockholm University}

\begin{document}
\begin{frame}{}
    \vfill
    \begin{columns}[t]
        \begin{column}{.48\linewidth}

            \begin{block}{\large Swedish Sign Language (SSL)}
                \begin{itemize}
                    \item Used by $\approx$ 10,000 as a primary language
                    \item Documented history of 200 years
                    \item Official status since 1981
                \end{itemize}
            \end{block}

            \begin{block}{\large Existing computer resources}
                \begin{itemize}
                    \item Swedish Sign Language Corpus (SSLC)
                    \item Swedish Sign Language Dictionary (SSLD)
                    \item (\textbf{NEW!}) Universal Dependencies treebank
                \end{itemize}
            \end{block}

            \begin{block}{\large Universal Dependencies}
                \begin{itemize}
                    \item Language-independent(ish) annotation standard
                    \item 70 treebanks in 50 languages
                    \item 49 spoken/written, (\textbf{NEW!}) 1 signed
                    \item 2 with transcribed conversation (Slovene, SSL)
                \end{itemize}
            \end{block}


        \end{column}

        \begin{column}{.48\linewidth}

            \begin{block}{\large The SSL Treebank}
                \begin{itemize}
                    \item A first step: 672 tokens in UD 1.4
                    \item Annotation work continues
                    \item The first dependency treebank of a sign language
                \end{itemize}
            \end{block}

            \begin{block}{\large Parseable?}
                \begin{itemize}
                    \item Well... 28 LAS (36 UAS) on the test set
                    \item 334 \emph{tokens} of training data is not enough for
                        serious parsing
                \end{itemize}
            \end{block}


            \begin{block}{\large Questions opened}
                \begin{itemize}
                    \item What is the sound of a tree falling where no one
                        hears it, or of one hand clapping?
                    \item (\textbf{NEW!}) What is the projectivity status of
                        a tree containing overlapping signs from two hands?
                    \item How can the UD formalism be better adapted to sign
                        languages?
                \end{itemize}
            \end{block}



           \begin{center}
            \includegraphics[width=1.0\linewidth]{../nodalida2017/treesizes.pdf}
           \end{center}


          \end{column}
        \end{columns}
    %\begin{block}{\Large Conclusions}
    %    \Large
    %\end{block}
           \begin{center}
\begin{dependency}[theme = default, label style={scale=1.0}]
   \begin{deptext}[column sep=1em]
      \texttt{verb} \& \texttt{verb} \& \texttt{verb} \& \texttt{noun} \& \texttt{det} \& \texttt{verb} \\
      \textsc{s{\"a}tta-sig} \& \textsc{{\"a}ta}(Q) \& \textsc{titta-p{\aa}} \& \textsc{sn{\"o}{\string^}gubbe} \& \textsc{pek} \& \textsc{{\"a}ta}(Q) \\
      \textsc{sit-down} \& \textsc{eat}(Q) \& \textsc{look-at} \& \textsc{snow{\string^}old-man} \& \textsc{point} \& \textsc{eat}(Q) \\
   \end{deptext}
   \deproot{1}{root}
   \depedge{1}{2}{conj}
   \depedge{1}{3}{conj}
   \depedge{3}{4}{dobj}
   \depedge{4}{5}{det}
   \depedge[arc angle=50]{2}{6}{conj}
   \node (translation) [below left of = \wordref{1}{1}, xshift=12cm, yshift=-3em] {
   `He is sitting there eating looking out at
   the snowman.'};
\end{dependency}
           \end{center}
      \end{frame}
    \end{document}

