%
% File nodalida2017.tex
%
% Contact beata.megyesi@lingfil.uu.se
%
% Based on the instruction file for Nodalida 2015 and EACL 2014
% which in turn was based on the instruction files for previous 
% ACL and EACL conferences.

\documentclass[11pt]{article}
\usepackage{nodalida2017}
%\usepackage{times}
\usepackage{mathptmx}
%\usepackage{txfonts}
\usepackage{url}
\usepackage{latexsym}
\special{papersize=210mm,297mm} % to avoid having to use "-t a4" with dvips 
%\setlength\titlebox{6.5cm}  % You can expand the title box if you really have to

\title{Universal Dependencies for Swedish Sign Language}

\author{First Author \\
  Affiliation / Address line 1 \\
  Affiliation / Address line 2 \\
  Affiliation / Address line 3 \\
  {\tt email@domain} \\\And
  Second Author \\
  Affiliation / Address line 1 \\
  Affiliation / Address line 2 \\
  Affiliation / Address line 3 \\
  {\tt email@domain} \\}

\date{}

\begin{document}
\maketitle
\begin{abstract}
    We describe the first effort to annotate a signed lanugage with syntactic
    dependency structure: the Swedish Sign Language portion of the
    Universal Dependencies treebanks. This is significant because the
    visual modality presents some unique challenges in analysis and
    annotation, such as the possibility of
    both hands articulating separate signs simultaneously.
    Our data is sourced from the Swedish Sign Language Corpus,
    and if used in conjunction these resources contain very richly
    annotated data: beyond syntactic annotations there are video files,
    a parallel text in written Swedish, signer metadata,
    and a variety of other annotations.
\end{abstract}

\section{Introduction}

The Universal Dependencies (UD) project \cite{Nivre2016ud} 
has produced an extensible but at its core language-independent standard for
morphological and syntactic annotation. This has been applied to the Universal
Dependencies treebanks \cite{ud14}, which in its latest release at the time of
writing (version 1.4) contains 64 treebanks in 47 languages---one of which is
Swedish Sign Language (SSL, ISO 639-3: \textsc{swl}), the topic of this
article.


\section{Universal Dependencies}

The Universal Dependencies project aims to provide uniform morphological and
syntactic (in the form of dependency trees) annotations across languages
\cite{Nivre2016ud}.\footnote{Note that our work predates version 2 of the UD
guidelines, and is based on the first version.}
Built on a language-universal common core of 17 parts of speech and
40 dependency relations, there are also language-specific guidelines which
interpret and when necessary extend those in the context of a given language.

\section{Swedish Sign Language}

Swedish Sign Language is ...

\section{Data source}

SSLC is ...
% TODO: se slutet av abstract, ge gärna några fler detaljer här och synka med
% abstract (= fixa mina misstag) om det behövs

\section{Annotation principles for SSL}

...
% TODO: sammanfatta de viktigaste språkspecifika fenomenen,
% gärna några exempelträd

\section{Treebank statistics}

The SSL treebank released in version 1.4 of the UD treebanks
contains 82 trees with a total of 672 sign tokens.
Annotation efforts are ongoing and we hope to publish an expanded version
of the treebank later this year, but the most significant contribution of
this work is that we have investigated the issues involved in dependency
annotation of a signed language.

% TODO: någon rolig liten undersökning?

\section{Conclusions and future work}

In releasing the Universal Dependencies treebank of Swedish Sign Language
(SSL), the first such resource for a signed language,
we hope to enable new computational research into sign language syntax.
We also hope to stimulate the development of Natural
Language Processing (NLP) tools capable of processing sign languages.
Finally, because we have both parallel data in Swedish and language-independent
syntactic annotations, we also believe this resource could prove particularly
useful in cross-lingual NLP.


%\section*{Acknowledgments}
%
% We wish to thank ...

\bibliographystyle{acl}
\bibliography{nodalida2017ud}

\end{document}
