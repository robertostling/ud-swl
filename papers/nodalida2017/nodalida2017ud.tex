%
% File nodalida2017.tex
%
% Contact beata.megyesi@lingfil.uu.se
%
% Based on the instruction file for Nodalida 2015 and EACL 2014
% which in turn was based on the instruction files for previous 
% ACL and EACL conferences.

\documentclass[11pt]{article}
\usepackage{nodalida2017}
%\usepackage{times}
\usepackage{mathptmx}
%\usepackage{txfonts}
\usepackage{url}
\usepackage{latexsym}
\usepackage{tikz-dependency}
\special{papersize=210mm,297mm} % to avoid having to use "-t a4" with dvips 
%\setlength\titlebox{6.5cm}  % You can expand the title box if you really have to

\title{Universal Dependencies for Swedish Sign Language}

\author{First Author \\
  Affiliation / Address line 1 \\
  Affiliation / Address line 2 \\
  Affiliation / Address line 3 \\
  {\tt email@domain} \\\And
  Second Author \\
  Affiliation / Address line 1 \\
  Affiliation / Address line 2 \\
  Affiliation / Address line 3 \\
  {\tt email@domain} \\}

\date{}

\begin{document}
\maketitle
\begin{abstract}
    We describe the first effort to annotate a signed language with syntactic
    dependency structure: the Swedish Sign Language portion of the
    Universal Dependencies treebanks. This is significant because the
    visual modality presents some unique challenges in analysis and
    annotation, such as the possibility of
    both hands articulating separate signs simultaneously.
    Our data is sourced from the Swedish Sign Language Corpus,
    and if used in conjunction these resources contain very richly
    annotated data: beyond syntactic annotations there are video files,
    a parallel text in written Swedish, signer metadata,
    and a variety of other annotations.
\end{abstract}

\section{Introduction}

The Universal Dependencies (UD) project \cite{Nivre2016ud} 
has produced an extensible but at its core language-independent standard for
morphological and syntactic annotation. This has been applied to the Universal
Dependencies treebanks \cite{ud14}, which in its latest release at the time of
writing (version 1.4) contains 64 treebanks in 47 languages---one of which is
Swedish Sign Language (SSL, ISO 639-3: \textsc{swl}), the topic of this
article.


\section{Universal Dependencies}

The Universal Dependencies project aims to provide uniform morphological and
syntactic (in the form of dependency trees) annotations across languages
\cite{Nivre2016ud}.\footnote{Note that our work predates version 2 of the UD
guidelines, and is based on the first version.}
Built on a language-universal common core of 17 parts of speech and
40 dependency relations, there are also language-specific guidelines which
interpret and when necessary extend those in the context of a given language.

\section{Swedish Sign Language}

Swedish Sign Language (SSL) is the main sign language of the Swedish Deaf community.\footnote{Capital D ``Deaf'' is generally used to refer to the language community as a cultural and linguistic group, rather than `deaf' as a medical label.} It is estimated to be used by at least 10,000 as one of their primary languages, and is the only sign language to be recognized in Swedish law, giving it a special status alongside the official minority languages \cite{Ahlgren2006sou,Parkvall2015siffror}. The history of SSL goes back at least 200 years, since the inauguration of the first Deaf school in Sweden, and has influenced the two sign languages of Finland (i.e.~Finnish Sign Language and Finland-Swedish Sign Language) with which SSL can be said to be related \cite{Bergman2010transmission}.

\section{Data source}

The SSL Corpus Project ran during the years 2009--2011 with the intention to establish the first systematically designed and publicly available corpus of SSL, resulting in the SSL Corpus (SSLC). Approximately 24 hours of video data of dyadic signing was recorded, comprising 42 signers of different age, gender, and geographical background, spanning 300 individual video files \cite{Mesch2012signed}. The translation and annotation work is still on-going, with new releases being available online as the work moves forward. The last official release of the SSLC includes just under 7 hours of video data \cite{Mesch2012dataset} along with annotation files containing 53,625 sign tokens across 6,197 sign types \cite{Mesch2016annotated}. 

The corpus is annotated using the ELAN software \cite{Wittenburg2006elan}, and the annotation files are distributed in the corresponding XML-based \texttt{.eaf} format. Each annotation file contains tiers on which annotations are aligned with the a video file, both video and annotation tiers being visible in the ELAN interface (see Figure~\ref{fig:sslc_elan}). The SSLC annotation files currently include tiers for sign glosses, and others for Swedish translations. Sign glosses are written word labels that represent signs with approximate meanings (e.g. \textsc{pro1} for a first person pronoun). The sign gloss annotation tiers are thus segmented for lexical items (i.e.~individual signs), and come in pairs for each signer---each tier representing one of the signer's hands \cite{Mesch2015gloss}. Sign glosses also contain a part-of-speech (PoS) tag which have been derived from manually correcting the output of a semi-automatic method for cross-lingual PoS tagging \cite{Ostling2015enriching}. The translation tier is segmented into longer chunks, representing stretches of discourse that can be represented by an idiomatic Swedish translation. However, the translation segmentations do not represent clausal boundaries in neither SSL nor Swedish \cite{Borstell2014segmenting}. More recently, a portion of the SSLC was segmented into clausal units and annotated for basic syntactic roles \cite{Borstell2016syntactic}, which led to the current UD annotation work. Figure~\ref{fig:sslc_elan} shows the basic view of the SSLC videos and annotations in the ELAN software, with tiers for sign glosses and translations on the video timeline.
% TODO: se slutet av abstract, ge gärna några fler detaljer här och synka med
% abstract (= fixa mina misstag) om det behövs

\begin{figure*}
	\centering
	\includegraphics[width=0.9\textwidth]{sslc_elan.png}
	\caption{Screenshot of an SSLC file in ELAN.}
	\label{fig:sslc_elan}
\end{figure*}

\section{Annotation procedure and principles for SSL}

The annotation of UD based syntactic structure started by coming up with a procedure for annotating a signed language using ELAN. Signed language is more simultaneous than spoken language, particularly in the use of paired parallel articulators in form of the signer's two hands \cite{Vermeerbergen2007simultaneity}. Because of this, and because the sign gloss annotation tiers are separate for the individual hands, we had to include one set of annotation tiers for each hand (i.e.~two per signer, or four per annotation file since there are two signers). We came up with an alignment system that combined the alignment with sign annotations and the linking of dependency relations between signs. Each tier set consisted of three tiers: \texttt{Index}; \texttt{Link}; and \texttt{UD}. The three tiers were generated using [[[technical stuff]]], creating dependent tiers to the sign gloss tiers. On the \texttt{Index} tier, each sign was assigned an integer based on its chronological position in the file (in case of overlapping signs, the onset was given precedence, or the dominant hand if onsets were identical)---this was done using a script iterating over all annotation files. On the \texttt{Link} tier, the index of the sign on which the sign was dependent was entered into each annotation cell. Lastly, on the \texttt{UD} tier, the dependency relation of a sign was entered into its annotation cell. Later, a custom script extracted this manually annotated data, and converted it into the CoNLL-U format used by the UD project. The method of using ELAN for linking and annotating dependency relations on parallel annotation tiers is shown in Figure~\ref{fig:sslc_elan_ud}.
% TODO: färre lågnivådetaljer, gör tydligt att/hur vi löser problemet med samtidig artikulation.

\begin{figure*}
	\centering
	\includegraphics[width=0.9\textwidth]{sslc_elan_ud.png}
	\caption{Screenshot of the UD annotation tiers and sign--dependency linking using ELAN.}
	\label{fig:sslc_elan_ud}
\end{figure*}

\begin{figure*}
\centering
\begin{dependency}[theme = simple]
   \begin{deptext}[column sep=1em]
      \texttt{verb} \& \texttt{verb} \& \texttt{verb} \& \texttt{noun} \& \texttt{det} \& \texttt{verb} \\
      \textsc{s{\"a}tta-sig} \& \textsc{{\"a}ta}(Q) \& \textsc{titta-p{\aa}} \& \textsc{sn{\"o}{\string^}gubbe} \& \textsc{pek} \& \textsc{{\"a}ta}(Q) \\
      \textsc{sit-down} \& \textsc{eat}(Q) \& \textsc{look-at} \& \textsc{snow{\string^}old-man} \& \textsc{point} \& \textsc{eat}(Q) \\
   \end{deptext}
   \deproot{1}{root}
   \depedge{1}{2}{conj}
   \depedge{1}{3}{conj}
   \depedge{3}{4}{dobj}
   \depedge{4}{5}{det}
   \depedge[arc angle=50]{2}{6}{conj}
\end{dependency}
\caption{A PoS and dependency tagged sentence from the SSLC.}
\label{fig:ssl_dep}
\end{figure*}
% TODO: sammanfatta de viktigaste språkspecifika fenomenen,
% gärna några exempelträd

\section{Treebank statistics}

The SSL treebank released in version 1.4 of the UD treebanks
contains 82 trees with a total of 672 sign tokens.
Annotation efforts are ongoing and we hope to publish an expanded version
of the treebank later this year, but the most significant contribution of
this work is that we have investigated the issues involved in dependency
annotation of a signed language.

% TODO: någon rolig liten undersökning?

\section{Conclusions and future work}

In releasing the Universal Dependencies treebank of Swedish Sign Language
(SSL), the first such resource for a signed language,
we hope to enable new computational research into sign language syntax.
We also hope to stimulate the development of Natural
Language Processing (NLP) tools capable of processing sign languages.
Finally, because we have both parallel data in Swedish and language-independent
syntactic annotations, we also believe this resource could prove particularly
useful in cross-lingual NLP.


%\section*{Acknowledgments}
%
% We wish to thank ...

\bibliographystyle{acl}
\bibliography{nodalida2017ud}

\end{document}
